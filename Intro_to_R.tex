\PassOptionsToPackage{unicode=true}{hyperref} % options for packages loaded elsewhere
\PassOptionsToPackage{hyphens}{url}
%
\documentclass[ignorenonframetext,]{beamer}
\usepackage{pgfpages}
\setbeamertemplate{caption}[numbered]
\setbeamertemplate{caption label separator}{: }
\setbeamercolor{caption name}{fg=normal text.fg}
\beamertemplatenavigationsymbolsempty
% Prevent slide breaks in the middle of a paragraph:
\widowpenalties 1 10000
\raggedbottom
\setbeamertemplate{part page}{
\centering
\begin{beamercolorbox}[sep=16pt,center]{part title}
  \usebeamerfont{part title}\insertpart\par
\end{beamercolorbox}
}
\setbeamertemplate{section page}{
\centering
\begin{beamercolorbox}[sep=12pt,center]{part title}
  \usebeamerfont{section title}\insertsection\par
\end{beamercolorbox}
}
\setbeamertemplate{subsection page}{
\centering
\begin{beamercolorbox}[sep=8pt,center]{part title}
  \usebeamerfont{subsection title}\insertsubsection\par
\end{beamercolorbox}
}
\AtBeginPart{
  \frame{\partpage}
}
\AtBeginSection{
  \ifbibliography
  \else
    \frame{\sectionpage}
  \fi
}
\AtBeginSubsection{
  \frame{\subsectionpage}
}
\usepackage{lmodern}
\usepackage{amssymb,amsmath}
\usepackage{ifxetex,ifluatex}
\usepackage{fixltx2e} % provides \textsubscript
\ifnum 0\ifxetex 1\fi\ifluatex 1\fi=0 % if pdftex
  \usepackage[T1]{fontenc}
  \usepackage[utf8]{inputenc}
  \usepackage{textcomp} % provides euro and other symbols
\else % if luatex or xelatex
  \usepackage{unicode-math}
  \defaultfontfeatures{Ligatures=TeX,Scale=MatchLowercase}
\fi
\usetheme[]{CambridgeUS}
% use upquote if available, for straight quotes in verbatim environments
\IfFileExists{upquote.sty}{\usepackage{upquote}}{}
% use microtype if available
\IfFileExists{microtype.sty}{%
\usepackage[]{microtype}
\UseMicrotypeSet[protrusion]{basicmath} % disable protrusion for tt fonts
}{}
\IfFileExists{parskip.sty}{%
\usepackage{parskip}
}{% else
\setlength{\parindent}{0pt}
\setlength{\parskip}{6pt plus 2pt minus 1pt}
}
\usepackage{hyperref}
\hypersetup{
            pdftitle={Introduction to R},
            pdfauthor={Bobbur Abhilash Chowdary; IIM Calcutta},
            pdfborder={0 0 0},
            breaklinks=true}
\urlstyle{same}  % don't use monospace font for urls
\newif\ifbibliography
\usepackage{color}
\usepackage{fancyvrb}
\newcommand{\VerbBar}{|}
\newcommand{\VERB}{\Verb[commandchars=\\\{\}]}
\DefineVerbatimEnvironment{Highlighting}{Verbatim}{commandchars=\\\{\}}
% Add ',fontsize=\small' for more characters per line
\usepackage{framed}
\definecolor{shadecolor}{RGB}{248,248,248}
\newenvironment{Shaded}{\begin{snugshade}}{\end{snugshade}}
\newcommand{\AlertTok}[1]{\textcolor[rgb]{0.94,0.16,0.16}{#1}}
\newcommand{\AnnotationTok}[1]{\textcolor[rgb]{0.56,0.35,0.01}{\textbf{\textit{#1}}}}
\newcommand{\AttributeTok}[1]{\textcolor[rgb]{0.77,0.63,0.00}{#1}}
\newcommand{\BaseNTok}[1]{\textcolor[rgb]{0.00,0.00,0.81}{#1}}
\newcommand{\BuiltInTok}[1]{#1}
\newcommand{\CharTok}[1]{\textcolor[rgb]{0.31,0.60,0.02}{#1}}
\newcommand{\CommentTok}[1]{\textcolor[rgb]{0.56,0.35,0.01}{\textit{#1}}}
\newcommand{\CommentVarTok}[1]{\textcolor[rgb]{0.56,0.35,0.01}{\textbf{\textit{#1}}}}
\newcommand{\ConstantTok}[1]{\textcolor[rgb]{0.00,0.00,0.00}{#1}}
\newcommand{\ControlFlowTok}[1]{\textcolor[rgb]{0.13,0.29,0.53}{\textbf{#1}}}
\newcommand{\DataTypeTok}[1]{\textcolor[rgb]{0.13,0.29,0.53}{#1}}
\newcommand{\DecValTok}[1]{\textcolor[rgb]{0.00,0.00,0.81}{#1}}
\newcommand{\DocumentationTok}[1]{\textcolor[rgb]{0.56,0.35,0.01}{\textbf{\textit{#1}}}}
\newcommand{\ErrorTok}[1]{\textcolor[rgb]{0.64,0.00,0.00}{\textbf{#1}}}
\newcommand{\ExtensionTok}[1]{#1}
\newcommand{\FloatTok}[1]{\textcolor[rgb]{0.00,0.00,0.81}{#1}}
\newcommand{\FunctionTok}[1]{\textcolor[rgb]{0.00,0.00,0.00}{#1}}
\newcommand{\ImportTok}[1]{#1}
\newcommand{\InformationTok}[1]{\textcolor[rgb]{0.56,0.35,0.01}{\textbf{\textit{#1}}}}
\newcommand{\KeywordTok}[1]{\textcolor[rgb]{0.13,0.29,0.53}{\textbf{#1}}}
\newcommand{\NormalTok}[1]{#1}
\newcommand{\OperatorTok}[1]{\textcolor[rgb]{0.81,0.36,0.00}{\textbf{#1}}}
\newcommand{\OtherTok}[1]{\textcolor[rgb]{0.56,0.35,0.01}{#1}}
\newcommand{\PreprocessorTok}[1]{\textcolor[rgb]{0.56,0.35,0.01}{\textit{#1}}}
\newcommand{\RegionMarkerTok}[1]{#1}
\newcommand{\SpecialCharTok}[1]{\textcolor[rgb]{0.00,0.00,0.00}{#1}}
\newcommand{\SpecialStringTok}[1]{\textcolor[rgb]{0.31,0.60,0.02}{#1}}
\newcommand{\StringTok}[1]{\textcolor[rgb]{0.31,0.60,0.02}{#1}}
\newcommand{\VariableTok}[1]{\textcolor[rgb]{0.00,0.00,0.00}{#1}}
\newcommand{\VerbatimStringTok}[1]{\textcolor[rgb]{0.31,0.60,0.02}{#1}}
\newcommand{\WarningTok}[1]{\textcolor[rgb]{0.56,0.35,0.01}{\textbf{\textit{#1}}}}
\usepackage{graphicx,grffile}
\makeatletter
\def\maxwidth{\ifdim\Gin@nat@width>\linewidth\linewidth\else\Gin@nat@width\fi}
\def\maxheight{\ifdim\Gin@nat@height>\textheight\textheight\else\Gin@nat@height\fi}
\makeatother
% Scale images if necessary, so that they will not overflow the page
% margins by default, and it is still possible to overwrite the defaults
% using explicit options in \includegraphics[width, height, ...]{}
\setkeys{Gin}{width=\maxwidth,height=\maxheight,keepaspectratio}
\setlength{\emergencystretch}{3em}  % prevent overfull lines
\providecommand{\tightlist}{%
  \setlength{\itemsep}{0pt}\setlength{\parskip}{0pt}}
\setcounter{secnumdepth}{0}

% set default figure placement to htbp
\makeatletter
\def\fps@figure{htbp}
\makeatother


\title{Introduction to R}
\author{Bobbur Abhilash Chowdary \and IIM Calcutta}
\date{August 2018}

\begin{document}
\frame{\titlepage}

\begin{frame}{Why R?}
\protect\hypertarget{why-r}{}

\begin{itemize}[<+->]
\tightlist
\item
  It's open source!
\item
  Need not run behind IT support to get the license renewed (SAS users?)
\item
  It's one of the fastest growing languages along with Python for
  Machine Learning and AI
\item
  Easy integration with GitHub and Latex
\item
  Reproducable research(?)
\item
  Imagine with one click your code analyzes your data and generates all
  the required tables \& graphs, and finally generates a nice pdf/word
  document which you can send it to a journal. This can be done in R.
\end{itemize}

\end{frame}

\begin{frame}[fragile]{Installing R}
\protect\hypertarget{installing-r}{}

\begin{itemize}
\item
  Download and install R from \url{https://www.r-project.org/}
\item
  User interface of just R is very bad. So we need to install R Studio.
  There are alternative to R-Studio like Emacs etc. You can also try
  them.
\item
  Download and install R-Studio from \url{https://www.rstudio.com/}.
  Free version is good enough.
\item
  R-studio runs on top of R. R-studio relies on R to execute all the
  commands. R-studio cannot function without R.
\item
  Don't change the order of installation. First R and then R-Studio.
  Otherwise sometime it can create problems.
\item
  Try this in console and check output to see if RStudio working
\end{itemize}

\begin{Shaded}
\begin{Highlighting}[]
\DecValTok{1}\OperatorTok{+}\DecValTok{2}
\end{Highlighting}
\end{Shaded}

\begin{verbatim}
[1] 3
\end{verbatim}

\end{frame}

\begin{frame}{Optional - Git -1}
\protect\hypertarget{optional---git--1}{}

Have you ever been frustated seeing files named Final\_Version,
New\_Final\_Version \ldots{}.? Have you ever wondered what is the
difference between Version\_1 and Version\_2?

Then what you need is a version control system. Git is a popular version
control system. Luckily R-Studio has very good interface with Git and
GitHub.

Once again Git is open source and free. One can download Git from
\url{https://git-scm.com/}.

Git Clients : Git \textasciitilde{} RStudio : R. There several Git
clients out there. You may use any of them (Sourcetree, GitUp, GitHub
etc).

Rstudio does most of the basic stuff a Git clients does. Did I forget to
say RStudio has good interface with Git?

\end{frame}

\begin{frame}{Optional - Git -2}
\protect\hypertarget{optional---git--2}{}

Also create GitHub account so you can store your code online. Private
repository facility is free if you register with IIM Cal id. GO through
the steps after installing Git.

Step 1 - Open Git Bash app and type the following commands

git config --global user.name `abcd efgh'\\
git config --global user.email
`\href{mailto:abcdexy@email.iimcal.ac.in}{\nolinkurl{abcdexy@email.iimcal.ac.in}}'\\
git config --global --list

Step 2 - Go to GitHub.com and create a new repository. Choose default
options. Open your repository to find readme.md file. Press ``clone or
download'' button on top right corner and copy the url.

Step 3 - Open Rstudio \textgreater{}File \textgreater{}New Project
\textgreater{}Version Control \textgreater{}Git . Paste the above url
and choose the folder you want to install the project in. A new folder
with repository name will be created.

\end{frame}

\begin{frame}{Optional - Git -3}
\protect\hypertarget{optional---git--3}{}

Step 4 - Open the readme.md file. Make some changes and save it. In your
Git pane in Rstudio (beside Environment pane) select Readme.md and press
commit. Type a commit message in new window and then press commit. Then
press push (arrow up) to send the commit to GitHub. Open GitHub.com to
find the changes reflected in Readme.md file there.

Use `diff' in Git pane to see changes you have made since last commit.
To revert to previous commits use the addin below the menu section in
RStudio. Git is most useful when you are colaborating with your
colleagues. You can copy the code from GitHub to your desktop

\end{frame}

\begin{frame}{RStudio Basics}
\protect\hypertarget{rstudio-basics}{}

`Source' pane shows the scripts. For a new script : RStudio
\textgreater{}File \textgreater{}New File \textgreater{}R Script

Script has .R extension. It is the place where you will write all your
code. Select the code and press `Run' option on top right corner to
execute it. Shortcut: CTRL + ENTER

The `Console' pane is where you will find the result of executing the
code you have written in script. Always read the output in console
especially if it is in \textbf{RED}

`Environment' pane is where you will find the datasets you have loaded
into the memory.

`Files' pane contains the list of files in the current working directory

\end{frame}

\begin{frame}[fragile]{Intro to R}
\protect\hypertarget{intro-to-r}{}

\begin{itemize}
\tightlist
\item
  Display current working directory and set current working directory
\end{itemize}

\begin{Shaded}
\begin{Highlighting}[]
\KeywordTok{getwd}\NormalTok{()}
\KeywordTok{setwd}\NormalTok{(}\StringTok{"G:/My Drive/Thesis/BGs"}\NormalTok{)}
\end{Highlighting}
\end{Shaded}

\begin{itemize}
\tightlist
\item
  Installing Packages and Loading Packages. Need to install a library
  only once per computer but have to load it everytime you restart.
\end{itemize}

\begin{Shaded}
\begin{Highlighting}[]
\KeywordTok{install.packages}\NormalTok{(}\StringTok{"data.table"}\NormalTok{)}
\KeywordTok{install.packages}\NormalTok{(}\StringTok{"tidyverse"}\NormalTok{)}
\KeywordTok{library}\NormalTok{(data.table)}
\KeywordTok{library}\NormalTok{(tidyverse)}
\end{Highlighting}
\end{Shaded}

\end{frame}

\begin{frame}[fragile]{Variable names}
\protect\hypertarget{variable-names}{}

\begin{itemize}
\item
  R is case sensitive. `Data' is not same as `data'.
\item
  Names can have ' . ' (dot), '\_', letters and numbers.
\item
  A good coding convention is to always start with a letter.
\item
  Press `ALT + SHIFT + K' for list of Keyboard shortcuts
\item
  whenever you find `+' in the console that implies R is waiting for you
  to complete your command. Try typing \texttt{1+} in console.
\item
  Always use `\textless{}-' for assigning. `=' also works but sometimes
  it won't.
\item
  Missing values in R are called `NA'. The only way to do conditional
  tests is ``is.na''.
\end{itemize}

\begin{Shaded}
\begin{Highlighting}[]
\NormalTok{x =}\DecValTok{3}\NormalTok{; y=}\DecValTok{4}\NormalTok{; x}\OperatorTok{==}\NormalTok{y}
\NormalTok{x =}\OtherTok{NA}\NormalTok{; y=}\OtherTok{NA}\NormalTok{; x}\OperatorTok{==}\NormalTok{y}
\NormalTok{x <-}\StringTok{ }\KeywordTok{c}\NormalTok{(}\DecValTok{1}\NormalTok{,}\DecValTok{2}\NormalTok{,}\DecValTok{3}\NormalTok{,}\DecValTok{4}\NormalTok{,}\DecValTok{5}\NormalTok{,}\DecValTok{6}\NormalTok{,}\OtherTok{NA}\NormalTok{); x}\OperatorTok{>}\DecValTok{3}\NormalTok{; x}\OperatorTok{<}\DecValTok{4}
\end{Highlighting}
\end{Shaded}

\#\#mtcars data set

mtcars is a default data set in R. We will use this extensively.

mtcars dataset details\\
{[}, 1{]} mpg Miles/(US) gallon\\
{[}, 2{]} cyl Number of cylinders\\
{[}, 3{]} disp Displacement (cu.in.)\\
{[}, 4{]} hp Gross horsepower\\
{[}, 5{]} drat Rear axle ratio\\
{[}, 6{]} wt Weight (1000 lbs)\\
{[}, 7{]} qsec 1/4 mile time\\
{[}, 8{]} vs Engine (0 = V-shaped, 1 = straight)\\
{[}, 9{]} am Transmission (0 = automatic, 1 = manual)\\
{[},10{]} gear Number of forward gears\\
{[},11{]} carb Number of carburetors

\#\#Checking out the data set

head() and tail() functions provide the first and last `n' lines of a
given dataset respectively . summary() function summarises and str()
gives an idea of structure of data set.

\begin{Shaded}
\begin{Highlighting}[]
\NormalTok{var1 <-}\StringTok{ }\KeywordTok{head}\NormalTok{(mtcars,}\DecValTok{10}\NormalTok{) }
\NormalTok{var2 <-}\StringTok{ }\KeywordTok{tail}\NormalTok{(ggplot2}\OperatorTok{::}\NormalTok{mpg,}\DecValTok{10}\NormalTok{)}
\KeywordTok{View}\NormalTok{(var1) }\CommentTok{# capital V in View}
\NormalTok{var1 <-}\StringTok{ }\KeywordTok{summary}\NormalTok{(mtcars)}
\KeywordTok{View}\NormalTok{(var1)}
\KeywordTok{str}\NormalTok{(mtcars)}
\end{Highlighting}
\end{Shaded}

Common types of variables.\\
int - integer; dbl - real number\\
chr - character vector; date - date\\
fctr - factor; lgl - logical

\end{frame}

\begin{frame}[fragile]{Column names}
\protect\hypertarget{column-names}{}

Finding Column names, renaming all column names and changing just one
column name

\begin{Shaded}
\begin{Highlighting}[]
\KeywordTok{colnames}\NormalTok{(var1)}
\KeywordTok{colnames}\NormalTok{(var1) <-}\StringTok{ }\KeywordTok{c}\NormalTok{(}\StringTok{"mpg"}\NormalTok{, }\StringTok{"cyl"}\NormalTok{, }\StringTok{"disp"}\NormalTok{, }\StringTok{"hp"}\NormalTok{, }\StringTok{"drat"}\NormalTok{, }
                \StringTok{"wt"}\NormalTok{, }\StringTok{"qsec"}\NormalTok{, }\StringTok{"vs"}\NormalTok{, }\StringTok{"am"}\NormalTok{, }\StringTok{"gear1"}\NormalTok{, }\StringTok{"carb"}\NormalTok{)}
\KeywordTok{setnames}\NormalTok{(var1, }\StringTok{"gear1"}\NormalTok{, }\StringTok{"gear"}\NormalTok{)}
\end{Highlighting}
\end{Shaded}

\end{frame}

\begin{frame}[fragile]{Cleaning Environment and Console}
\protect\hypertarget{cleaning-environment-and-console}{}

\begin{itemize}
\tightlist
\item
  Removing a specific data set
\end{itemize}

\begin{Shaded}
\begin{Highlighting}[]
\KeywordTok{rm}\NormalTok{(var1, var2)}
\end{Highlighting}
\end{Shaded}

\begin{itemize}
\tightlist
\item
  Removing all data sets
\end{itemize}

\begin{Shaded}
\begin{Highlighting}[]
\KeywordTok{rm}\NormalTok{(}\DataTypeTok{list =} \KeywordTok{ls}\NormalTok{())}
\end{Highlighting}
\end{Shaded}

\begin{itemize}
\tightlist
\item
  Clear Console
\end{itemize}

\begin{Shaded}
\begin{Highlighting}[]
\KeywordTok{cat}\NormalTok{(}\StringTok{"}\CharTok{\textbackslash{}014}\StringTok{"}\NormalTok{)}
\end{Highlighting}
\end{Shaded}

\end{frame}

\begin{frame}[fragile]{Reading and writing files}
\protect\hypertarget{reading-and-writing-files}{}

\begin{Shaded}
\begin{Highlighting}[]
\NormalTok{var0 <-}\StringTok{ }\NormalTok{mtcars}
\KeywordTok{write.csv}\NormalTok{(var0, }\DataTypeTok{file =} \StringTok{"Output/data.csv"}\NormalTok{)}
\KeywordTok{write.table}\NormalTok{(var0, }\DataTypeTok{file =} \StringTok{"Output/data1.txt"}\NormalTok{, }\DataTypeTok{sep =} \StringTok{"|"}\NormalTok{)}
\NormalTok{data <-}\StringTok{ }\KeywordTok{read.csv}\NormalTok{(}\DataTypeTok{file =} \StringTok{"Output/data.csv"}\NormalTok{, }\DataTypeTok{row.names =} \DecValTok{1}\NormalTok{)}
\NormalTok{data1 <-}\StringTok{ }\KeywordTok{read.table}\NormalTok{(}\StringTok{"Output/data1.txt"}\NormalTok{, }
                    \DataTypeTok{header =} \OtherTok{TRUE}\NormalTok{, }\DataTypeTok{sep =} \StringTok{"|"}\NormalTok{)}
\NormalTok{data2 <-}\StringTok{ }\KeywordTok{read.csv}\NormalTok{(}\DataTypeTok{file =} \StringTok{"Output/data.csv"}\NormalTok{)}
\KeywordTok{download.file}\NormalTok{(}\StringTok{"https://www.nseindia.com/content/historical/EQUITIES/2018/JUL/cm30JUL2018bhav.csv.zip"}\NormalTok{,}
              \StringTok{"NSE BhavCopy/cm30JUL2018bhav.csv.zip"}\NormalTok{)}
\end{Highlighting}
\end{Shaded}

For better import funtions refer `readr' package. Use `readxl' package
for reading Excel data sheets. Use `haven' \& `foreign' packages for
reading SAS, STATA and SPSS files.

\end{frame}

\begin{frame}{Basic data operations}
\protect\hypertarget{basic-data-operations}{}

Most operations in data cleaning process fall into these categories

filter - selecting observations/rows\\
arrange - ordering the observations\\
select - selecting columns\\
mutate - creating new variables from existing ones\\
summarize - summary

The above five are functions in `dplyr' package. `dplyr' and a couple of
other packages together are called `tidyverse' which we already loaded.

\end{frame}

\begin{frame}[fragile]{dplyr - filter}
\protect\hypertarget{dplyr---filter}{}

Use `==' when its a test/question. Use `=' for assigning/informing. We
use `==' below because we want to ask ``if gear = 4 then select'' which
is a test.

\begin{Shaded}
\begin{Highlighting}[]
\NormalTok{mtcars1 <-}\StringTok{ }\KeywordTok{filter}\NormalTok{(mtcars, gear }\OperatorTok{==}\StringTok{ }\DecValTok{4}\NormalTok{, cyl }\OperatorTok{==}\StringTok{ }\DecValTok{6}\NormalTok{)}
\NormalTok{mtcars2 <-}\StringTok{ }\KeywordTok{filter}\NormalTok{(mtcars, gear }\OperatorTok{==}\StringTok{ }\DecValTok{4} \OperatorTok{&}\StringTok{ }\NormalTok{cyl }\OperatorTok{==}\StringTok{ }\DecValTok{6}\NormalTok{) }\CommentTok{# =mtcars1}
\NormalTok{mtcars3 <-}\StringTok{ }\KeywordTok{filter}\NormalTok{(mtcars, gear }\OperatorTok{==}\StringTok{ }\DecValTok{4} \OperatorTok{|}\StringTok{ }\NormalTok{cyl }\OperatorTok{==}\StringTok{ }\DecValTok{6}\NormalTok{) }\CommentTok{#OR operator}
\NormalTok{mtcars4 <-}\StringTok{ }\KeywordTok{filter}\NormalTok{(mtcars, gear }\OperatorTok{==}\StringTok{ }\DecValTok{4} \OperatorTok{|}\StringTok{ }\NormalTok{gear }\OperatorTok{==}\StringTok{ }\DecValTok{6}\NormalTok{)}
\NormalTok{mtcars5 <-}\StringTok{ }\KeywordTok{filter}\NormalTok{(mtcars, gear }\OperatorTok\StringTok{ }\KeywordTok{c}\NormalTok{(}\DecValTok{4}\NormalTok{,}\DecValTok{6}\NormalTok{)) }\CommentTok{# =mtcars4}
\end{Highlighting}
\end{Shaded}

All `dplyr operations dont affect the original dataset. mtcars still has
32 observations. Try 'mtcars' on the left side of the above equation.
Common operators recognized in R - ``\textgreater{}, \textless{},
\textgreater{}=, \textless{}=, != (not equal) and ==''

\end{frame}

\begin{frame}[fragile]{dplyr - Arrange and select}
\protect\hypertarget{dplyr---arrange-and-select}{}

Similar to filter(), first input to arrange() is dataset name followed
by column names for sorting. use desc() for descing order.

\begin{Shaded}
\begin{Highlighting}[]
\KeywordTok{arrange}\NormalTok{(mtcars, mpg, }\KeywordTok{desc}\NormalTok{(cyl))}
\end{Highlighting}
\end{Shaded}

Selecting columns uses select() function which is similar to arrange()
and filter() functions

\begin{Shaded}
\begin{Highlighting}[]
\KeywordTok{select}\NormalTok{(mtcars, mpg, cyl)}
\KeywordTok{select}\NormalTok{(mtcars, }\OperatorTok{-}\NormalTok{mpg)}
\KeywordTok{select}\NormalTok{(mtcars, vs, }\KeywordTok{everything}\NormalTok{()) }\CommentTok{#rearranging}
\end{Highlighting}
\end{Shaded}

\end{frame}

\begin{frame}[fragile]{dplyr - creating variables}
\protect\hypertarget{dplyr---creating-variables}{}

Similarly use `mutate' for creating new variables

\begin{Shaded}
\begin{Highlighting}[]
\KeywordTok{mutate}\NormalTok{(mtcars, }\DataTypeTok{mpg_per_hp =}\NormalTok{ mpg}\OperatorTok{/}\NormalTok{hp, }\DataTypeTok{mpg_per_wt =}\NormalTok{ mpg}\OperatorTok{/}\NormalTok{wt, }
       \DataTypeTok{mpg_per_wt1 =}\NormalTok{ mpg_per_hp}\OperatorTok{*}\NormalTok{hp}\OperatorTok{/}\NormalTok{wt)}
\KeywordTok{transmutate}\NormalTok{(mtcars, }\DataTypeTok{mpg_per_hp =}\NormalTok{ mpg}\OperatorTok{/}\NormalTok{hp, }\DataTypeTok{mpg_per_wt =}\NormalTok{ mpg}\OperatorTok{/}\NormalTok{wt, }
       \DataTypeTok{mpg_per_wt1 =}\NormalTok{ mpg_per_hp}\OperatorTok{*}\NormalTok{hp}\OperatorTok{/}\NormalTok{wt)}
\end{Highlighting}
\end{Shaded}

You can use the following operators in mutate - '+, -, /, *, \^{},\\
\%\% (reminder e.g. - 5 \%\% 2 =1), \%/\% (e.g. - 5 \%/\% 2 = 2), sum(),
cumsum(), prod()

\end{frame}

\begin{frame}[fragile]{dplyr - summarize() and Pipes}
\protect\hypertarget{dplyr---summarize-and-pipes}{}

\begin{Shaded}
\begin{Highlighting}[]
\NormalTok{by_gear <-}\StringTok{ }\KeywordTok{group_by}\NormalTok{(mtcars, gear, carb)}
\KeywordTok{summarise}\NormalTok{(by_gear, }\DataTypeTok{mean_milage =} \KeywordTok{mean}\NormalTok{(mpg))}
\end{Highlighting}
\end{Shaded}

Pipes are very useful to write good looking code. The most used Pipe is
\%\textgreater{}\%. Pipes also reduce the need to save intermediaries.

x \%\textgreater{}\% f(y) is equvalent to f(x,y). Above example can be
rewritten as\ldots{}

\begin{Shaded}
\begin{Highlighting}[]
\NormalTok{mtcars }\OperatorTok\StringTok{ }\KeywordTok{group_by}\NormalTok{(gear, carb) }\OperatorTok\StringTok{ }
\StringTok{  }\KeywordTok{summarise}\NormalTok{(}\DataTypeTok{mean_milage =} \KeywordTok{mean}\NormalTok{(mpg))}
\end{Highlighting}
\end{Shaded}

\begin{Shaded}
\begin{Highlighting}[]
\KeywordTok{ungroup}\NormalTok{(mtcars)}
\end{Highlighting}
\end{Shaded}

\end{frame}

\begin{frame}[fragile]{ggplot}
\protect\hypertarget{ggplot}{}

\begin{Shaded}
\begin{Highlighting}[]
\KeywordTok{ggplot}\NormalTok{(mtcars ) }\OperatorTok{+}\StringTok{ }\KeywordTok{geom_point}\NormalTok{(}\KeywordTok{aes}\NormalTok{(}\DataTypeTok{x=}\NormalTok{ disp, }\DataTypeTok{y =}\NormalTok{ mpg))}
\end{Highlighting}
\end{Shaded}

\includegraphics{Intro_to_R_files/figure-beamer/unnamed-chunk-17-1.pdf}

\end{frame}

\begin{frame}[fragile]

\begin{Shaded}
\begin{Highlighting}[]
\KeywordTok{ggplot}\NormalTok{(mtcars, }\KeywordTok{aes}\NormalTok{(}\DataTypeTok{x=}\NormalTok{ disp, }\DataTypeTok{y =}\NormalTok{ mpg))}\OperatorTok{+}\KeywordTok{geom_point}\NormalTok{(}\KeywordTok{aes}\NormalTok{(}
  \DataTypeTok{size =}\NormalTok{ hp,}\DataTypeTok{color =}\NormalTok{ gear)) }\OperatorTok{+}\StringTok{ }\KeywordTok{geom_smooth}\NormalTok{()}
\end{Highlighting}
\end{Shaded}

\begin{center}\includegraphics{Intro_to_R_files/figure-beamer/unnamed-chunk-18-1} \end{center}

\end{frame}

\begin{frame}[fragile]

\begin{Shaded}
\begin{Highlighting}[]
\KeywordTok{ggplot}\NormalTok{(mtcars)}\OperatorTok{+}\StringTok{ }\KeywordTok{geom_bar}\NormalTok{(}\DataTypeTok{mapping =} \KeywordTok{aes}\NormalTok{(}\DataTypeTok{x =}\NormalTok{ gear))}
\end{Highlighting}
\end{Shaded}

\begin{center}\includegraphics{Intro_to_R_files/figure-beamer/unnamed-chunk-19-1} \end{center}

\end{frame}

\begin{frame}[fragile]

\begin{Shaded}
\begin{Highlighting}[]
\KeywordTok{ggplot}\NormalTok{(mtcars)}\OperatorTok{+}\StringTok{ }\KeywordTok{geom_histogram}\NormalTok{(}\DataTypeTok{mapping=}\KeywordTok{aes}\NormalTok{(}\DataTypeTok{x=}\NormalTok{hp),}\DataTypeTok{bins=}\DecValTok{15}\NormalTok{)}
\end{Highlighting}
\end{Shaded}

\begin{center}\includegraphics{Intro_to_R_files/figure-beamer/unnamed-chunk-20-1} \end{center}

\end{frame}

\begin{frame}[fragile]{Transforming data (tidyr package) - creating
dataset}
\protect\hypertarget{transforming-data-tidyr-package---creating-dataset}{}

\begin{Shaded}
\begin{Highlighting}[]
\NormalTok{data <-}\StringTok{ }\KeywordTok{data.frame}\NormalTok{(}\DataTypeTok{alpha =} \KeywordTok{rep}\NormalTok{(LETTERS[}\DecValTok{1}\OperatorTok{:}\DecValTok{3}\NormalTok{], }\DecValTok{2}\NormalTok{), }
\DataTypeTok{name =} \KeywordTok{sort}\NormalTok{(}\KeywordTok{rep}\NormalTok{(}\KeywordTok{c}\NormalTok{(}\StringTok{"bobbur"}\NormalTok{,}\StringTok{"abhi"}\NormalTok{),}\DecValTok{3}\NormalTok{)), }\DataTypeTok{year=}\DecValTok{2011}\OperatorTok{:}\DecValTok{2016}\NormalTok{)}
\end{Highlighting}
\end{Shaded}

\begin{verbatim}
  alpha   name year
1     A   abhi 2011
2     B   abhi 2012
3     C   abhi 2013
4     A bobbur 2014
5     B bobbur 2015
6     C bobbur 2016
\end{verbatim}

\end{frame}

\begin{frame}[fragile]{Transforming data - spread()}
\protect\hypertarget{transforming-data---spread}{}

\begin{Shaded}
\begin{Highlighting}[]
\KeywordTok{spread}\NormalTok{(data, }\DataTypeTok{key =}\NormalTok{ name, }\DataTypeTok{value =}\NormalTok{ year) }
\end{Highlighting}
\end{Shaded}

\begin{verbatim}
  alpha abhi bobbur
1     A 2011   2014
2     B 2012   2015
3     C 2013   2016
\end{verbatim}

\end{frame}

\begin{frame}[fragile]{Transforming data - gather()}
\protect\hypertarget{transforming-data---gather}{}

\begin{Shaded}
\begin{Highlighting}[]
\KeywordTok{gather}\NormalTok{(data, abhi, bobbur, }\DataTypeTok{key =}\NormalTok{ name, }\DataTypeTok{value =}\NormalTok{ year)}
\end{Highlighting}
\end{Shaded}

\begin{verbatim}
  alpha   name year
1     A   abhi 2011
2     B   abhi 2012
3     C   abhi 2013
4     A bobbur 2014
5     B bobbur 2015
6     C bobbur 2016
\end{verbatim}

\end{frame}

\begin{frame}[fragile]{Others - Joining datasets - rbind}
\protect\hypertarget{others---joining-datasets---rbind}{}

\begin{Shaded}
\begin{Highlighting}[]
\NormalTok{data1 <-}\StringTok{ }\NormalTok{data}
\NormalTok{data1}\OperatorTok{$}\NormalTok{year <-}\StringTok{ }\NormalTok{data1}\OperatorTok{$}\NormalTok{year }\OperatorTok{+}\StringTok{ }\DecValTok{1000}
\NormalTok{data2 <-}\StringTok{ }\KeywordTok{rbind}\NormalTok{(data, data1)}
\KeywordTok{head}\NormalTok{(data2,}\DecValTok{8}\NormalTok{)}
\end{Highlighting}
\end{Shaded}

\begin{verbatim}
  alpha   name year
1     A   abhi 2011
2     B   abhi 2012
3     C   abhi 2013
4     A bobbur 2014
5     B bobbur 2015
6     C bobbur 2016
7     A   abhi 3011
8     B   abhi 3012
\end{verbatim}

\end{frame}

\begin{frame}[fragile]{Joining datasets - merge}
\protect\hypertarget{joining-datasets---merge}{}

\begin{Shaded}
\begin{Highlighting}[]
\KeywordTok{merge}\NormalTok{(data,data1, }\DataTypeTok{by =} \KeywordTok{c}\NormalTok{(}\StringTok{"alpha"}\NormalTok{,}\StringTok{"name"}\NormalTok{))}
\end{Highlighting}
\end{Shaded}

\begin{verbatim}
  alpha   name year.x year.y
1     A   abhi   2011   3011
2     A bobbur   2014   3014
3     B   abhi   2012   3012
4     B bobbur   2015   3015
5     C   abhi   2013   3013
6     C bobbur   2016   3016
\end{verbatim}

\end{frame}

\begin{frame}[fragile]{IF ELSE, IFELSE and for loops}
\protect\hypertarget{if-else-ifelse-and-for-loops}{}

\begin{Shaded}
\begin{Highlighting}[]
\NormalTok{x=}\DecValTok{2}
\ControlFlowTok{if}\NormalTok{(x}\OperatorTok{==}\DecValTok{1}\NormalTok{)\{}\KeywordTok{print}\NormalTok{(}\StringTok{"hello World"}\NormalTok{) \} }\ControlFlowTok{else} \ControlFlowTok{if}\NormalTok{(x}\OperatorTok{==}\DecValTok{2}\NormalTok{) }
\NormalTok{  \{}\KeywordTok{print}\NormalTok{(}\StringTok{"hello!"}\NormalTok{)\} }\ControlFlowTok{else}\NormalTok{ \{}\KeywordTok{print}\NormalTok{(}\StringTok{"!"}\NormalTok{)\}}
\NormalTok{data1}\OperatorTok{$}\NormalTok{dummy <-}\StringTok{ }\KeywordTok{ifelse}\NormalTok{(data1}\OperatorTok{$}\NormalTok{year}\OperatorTok{>}\DecValTok{3012}\NormalTok{, }\DecValTok{1}\NormalTok{, }\DecValTok{0}\NormalTok{)}
\ControlFlowTok{for}\NormalTok{ (i }\ControlFlowTok{in} \DecValTok{1}\OperatorTok{:}\KeywordTok{nrow}\NormalTok{(data1))\{data1}\OperatorTok{$}\NormalTok{dummy1[i] <-}\StringTok{ }
\StringTok{                        }\NormalTok{data1}\OperatorTok{$}\NormalTok{dummy[i]}\OperatorTok{+}\DecValTok{1}\NormalTok{ \}}
\NormalTok{data1}\OperatorTok{$}\NormalTok{name <-}\StringTok{ }\OtherTok{NULL} 
\end{Highlighting}
\end{Shaded}

\begin{verbatim}
[1] "hello!"
\end{verbatim}

\begin{verbatim}
  alpha year dummy dummy1
1     A 3011     0      1
2     B 3012     0      1
3     C 3013     1      2
\end{verbatim}

\end{frame}

\begin{frame}[fragile]{Row Selection}
\protect\hypertarget{row-selection}{}

\begin{Shaded}
\begin{Highlighting}[]
\NormalTok{data4 <-}\StringTok{ }\KeywordTok{data.frame}\NormalTok{(}\DataTypeTok{alpha =} \KeywordTok{rep}\NormalTok{(LETTERS[}\DecValTok{1}\OperatorTok{:}\DecValTok{3}\NormalTok{], }\DecValTok{2}\NormalTok{), }\DataTypeTok{num =} \DecValTok{1}\OperatorTok{:}\DecValTok{6}\NormalTok{)}
\NormalTok{elim <-}\StringTok{ }\KeywordTok{c}\NormalTok{(}\OtherTok{TRUE}\NormalTok{, }\OtherTok{FALSE}\NormalTok{, }\OtherTok{TRUE}\NormalTok{, }\OtherTok{TRUE}\NormalTok{, }\OtherTok{FALSE}\NormalTok{, }\OtherTok{FALSE}\NormalTok{)}
\NormalTok{data4[elim,]  }
\NormalTok{data4[num }\OperatorTok{>}\StringTok{ }\DecValTok{3}\NormalTok{,]  }
\NormalTok{data4[data4}\OperatorTok{$}\NormalTok{num }\OperatorTok{>}\DecValTok{3}\NormalTok{,]  }
\end{Highlighting}
\end{Shaded}

\begin{verbatim}
  alpha num beta
1     A   1    b
3     C   3    b
4     A   4    a
\end{verbatim}

\end{frame}

\begin{frame}[fragile]{Row selection}
\protect\hypertarget{row-selection-1}{}

\begin{Shaded}
\begin{Highlighting}[]
\NormalTok{data4[data4}\OperatorTok{$}\NormalTok{num }\OperatorTok{>}\DecValTok{3}\NormalTok{,]}
\NormalTok{data4[}\KeywordTok{c}\NormalTok{(}\DecValTok{1}\NormalTok{,}\DecValTok{2}\NormalTok{,}\DecValTok{2}\NormalTok{,}\DecValTok{2}\NormalTok{),]}
\end{Highlighting}
\end{Shaded}

\begin{verbatim}
  alpha num beta
4     A   4    a
5     B   5    a
6     C   6    a
    alpha num beta
1       A   1    b
2       B   2    b
2.1     B   2    b
2.2     B   2    b
\end{verbatim}

\end{frame}

\begin{frame}[fragile]{Selection of columns}
\protect\hypertarget{selection-of-columns}{}

\begin{Shaded}
\begin{Highlighting}[]
\NormalTok{data4[,}\KeywordTok{c}\NormalTok{(}\DecValTok{2}\NormalTok{,}\DecValTok{3}\NormalTok{)] }\OperatorTok\StringTok{ }\KeywordTok{head}\NormalTok{(}\DataTypeTok{n=}\DecValTok{3}\NormalTok{)}
\NormalTok{data4[,}\KeywordTok{c}\NormalTok{(}\StringTok{'num'}\NormalTok{, }\StringTok{'alpha'}\NormalTok{)] }\OperatorTok\StringTok{ }\KeywordTok{head}\NormalTok{(}\DataTypeTok{n=}\DecValTok{3}\NormalTok{)}
\end{Highlighting}
\end{Shaded}

\begin{verbatim}
  num beta
1   1    b
2   2    b
3   3    b
  num alpha
1   1     A
2   2     B
3   3     C
\end{verbatim}

\end{frame}

\begin{frame}[fragile]{duplicates}
\protect\hypertarget{duplicates}{}

\begin{Shaded}
\begin{Highlighting}[]
\NormalTok{data4[}\KeywordTok{duplicated}\NormalTok{(data4}\OperatorTok{$}\NormalTok{alpha),]}
\NormalTok{data4[}\OperatorTok{!}\KeywordTok{duplicated}\NormalTok{(data4}\OperatorTok{$}\NormalTok{beta),]}
\end{Highlighting}
\end{Shaded}

\begin{verbatim}
  alpha num beta
4     A   4    a
5     B   5    a
6     C   6    a
  alpha num beta
1     A   1    b
4     A   4    a
\end{verbatim}

\end{frame}

\begin{frame}[fragile]{order}
\protect\hypertarget{order}{}

\begin{Shaded}
\begin{Highlighting}[]
\NormalTok{data4[}\KeywordTok{order}\NormalTok{(data4}\OperatorTok{$}\NormalTok{alpha),]  }\OperatorTok\StringTok{ }\KeywordTok{head}\NormalTok{(}\DataTypeTok{n=}\DecValTok{3}\NormalTok{)}
\NormalTok{data4[}\KeywordTok{order}\NormalTok{(data4}\OperatorTok{$}\NormalTok{alpha, data4}\OperatorTok{$}\NormalTok{beta),] }\OperatorTok\StringTok{ }\KeywordTok{head}\NormalTok{(}\DataTypeTok{n=}\DecValTok{3}\NormalTok{)}
\end{Highlighting}
\end{Shaded}

\begin{verbatim}
  alpha num beta
1     A   1    b
4     A   4    a
2     B   2    b
  alpha num beta
4     A   4    a
1     A   1    b
5     B   5    a
\end{verbatim}

\end{frame}

\begin{frame}

Thank you!

\end{frame}

\end{document}
